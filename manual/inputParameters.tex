\chapter{Input Parameters}
\label{ch:input}

\newcommand{\param}[5]{{\setlength{\parindent}{0cm} {\ttfamily \bfseries \hypertarget{#1}{#1}}\\{\it Type}: #2\\{\it Default}: #3\\{\it When it matters}: #4\\{\it Meaning}: #5}}
\newcommand{\myhrule}{{\setlength{\parindent}{0cm} \hrulefill }}

\newcommand{\true}{{\ttfamily .true.}}
\newcommand{\false}{{\ttfamily .false.}}

In this section we describe all the parameters which can be included in the input namelist. 

\section{General parameters}

\param{symmetry\_option}
{integer}
{1}
{Always}
{Determines whether stellarator symmetry is imposed.\\

{\ttfamily symmetry\_option} = 1: Force the magnetization to be stellarator symmetric.\\

{\ttfamily symmetry\_option} = 2: Force the magnetization to be stellarator-symmetric, and a test will be done to verify symmetry of the inductance matrix, which adds some time to program execution.\\

{\ttfamily symmetry\_option} = 3: No symmetry in the magnetization is imposed.
}

\myhrule

\param{save\_level}
{integer}
{3}
{Always}
{Option related determining how many variables are saved in the \netCDF~output file.  The larger the value, the smaller the output file.\\

{\ttfamily save\_level} = 0: Save everything.\\

{\ttfamily save\_level} = 1: Do not save the matrix $g$ that relates each degree of freedom of $\vect{M}$ to $B_{normal}$ at each point on the plasma surface.\\

{\ttfamily save\_level} = 3: Also do not save the plasma normal vector or derivatives of the position vector, or the matrices representing $\chi_B^2$ or $\chi_M^2$.
}

\myhrule

\section{Resolution parameters}

For any new set of surface geometries you consider, you should vary the resolution parameters in this section to make sure
they are large enough.  These parameters should be large enough that the code results you care about are unchanged under further
resolution increases. (A possible exception is \parlink{ns\_magnetization}, which you may want to set to 1 to make the magnetization
uniform in the $s$ direction.)

\myhrule

\param{ntheta\_plasma}
{integer}
{64}
{Always}
{Number of grid points in poloidal angle used to evaluate surface integrals on the plasma surface.
Often 64 or 128 is a good value.
It is resonable and common but not mandatory to use the same value for {\ttfamily ntheta\_plasma} and \parlink{ntheta\_coil}.}

\myhrule

\param{ntheta\_coil}
{integer}
{64}
{Always}
{Number of grid points in poloidal angle used to evaluate integrals over the magnetization region.
Often 64 or 128 is a good value.
It is resonable and common but not mandatory to use the same value for \parlink{ntheta\_plasma} and {\ttfamily ntheta\_coil}.}

\myhrule


\param{nzeta\_plasma}
{integer}
{64}
{Always}
{Number of grid points in toroidal angle used to evaluate surface integrals on the plasma surface.
Often 64 or 128 is a good value.
It is resonable and common but not mandatory to use the same value for {\ttfamily nzeta\_plasma} and \parlink{nzeta\_coil}.}

\myhrule

\param{nzeta\_coil}
{integer}
{64}
{Always}
{Number of grid points in toroidal angle used to evaluate integrals over the magnetization region.
Often 64 or 128 is a good value.
It is resonable and common but not mandatory to use the same value for \parlink{nzeta\_plasma} and {\ttfamily nzeta\_coil}.}

\myhrule

\param{mpol\_magnetization}
{integer}
{12}
{Always}
{Maximum poloidal mode number to include for the magnetization.
}

\myhrule

\param{ntor\_magnetization}
{integer}
{12}
{Always}
{
Maximum toroidal mode number to include for the magnetization.
}

\myhrule

\param{ns\_integration}
{integer}
{2}
{Always}
{
Number of grid points in the radial coordinate $s$ used for integration over the magnetization region.
It is required that {\ttfamily ns\_integration} $\ge$ \parlink{ns\_magnetization}.
}

\myhrule

\param{ns\_magnetization}
{integer}
{1}
{Always}
{
Number of grid points in the radial coordinate $s$ used to discretize the magnetization vector. If you wish the Cartesian components of $\vect{M}$ to be
independent of $s$, you can set {\ttfamily ns\_magnetization=1}.
}

\myhrule

\param{mpol\_transform\_refinement}
{real}
{5.0}
{Only when \parlink{geometry\_option\_plasma} is 4.}
{The number of poloidal mode numbers in the \vmec~file will be multiplied by this value
when transforming from the original poloidal angle to the straight-field-line angle.
Since the original \vmec~angle is chosen to minimize the number of Fourier modes required,
more modes are required in any other coordinate.
This parameter affects the time required to compute constant-offset surfaces,
but does not affect the time for other calculations.
}

\myhrule

\param{ntor\_transform\_refinement}
{real}
{1.0}
{Only when \parlink{geometry\_option\_plasma} is 4.}
{The number of toroidal mode numbers in the \vmec~file will be multiplied by this value
when transforming from the original poloidal angle to the straight-field-line angle.
Since the original \vmec~angle is chosen to minimize the number of Fourier modes required,
more modes are required in any other coordinate.
This parameter affects the time required to compute constant-offset surfaces,
but does not affect the time for other calculations.
}

\myhrule

\section{Geometry parameters for the plasma surface}

\param{geometry\_option\_plasma}
{integer}
{0}
{Always}
{This option controls how you specify the shape of the target plasma surface.\\

{\ttfamily geometry\_option\_plasma} = 0: The plasma surface will be a plain circular torus. The major radius will be \parlink{R0\_plasma}.
     The minor radius will be \parlink{a\_plasma}. This option exists just for testing purposes.\\

{\ttfamily geometry\_option\_plasma} = 1: Identical to option 0.\\

{\ttfamily geometry\_option\_plasma} = 2: The plasma surface will be the last surface in the full radial grid of the \vmec~file specified by \parlink{wout\_filename}.
The poloidal angle used will be the normal \vmec~angle which is not a straight-field-line coordinate.
This is typically the best option to use for working with \vmec~equilibria.\\

{\ttfamily geometry\_option\_plasma} = 3: The plasma surface will be the last surface in the half radial grid of the \vmec~file specified by \parlink{wout\_filename}.
The poloidal angle used will be the normal \vmec~angle which is not a straight-field-line coordinate.
This option exists so that the same flux surface can be used when comparing with {\ttfamily geometry\_option\_plasma} = 4.\\

{\ttfamily geometry\_option\_plasma} = 4: The plasma surface will be the last surface in the half radial grid of the \vmec~file specified by \parlink{wout\_filename}.
The poloidal angle used will be the straight-field-line coordinate, obtained by shifting the normal \vmec~poloidal angle by \vmec's $\lambda$ quantity.
This option exists in order to examine changes when using a different poloidal coordinate compared to {\ttfamily geometry\_option\_plasma} = 3.\\

{\ttfamily geometry\_option\_plasma} = 5: The plasma surface will be the flux surface with normalized poloidal flux
\parlink{efit\_psiN} taken from the {\ttfamily efit} file specified by \parlink{efit\_filename}. \\

{\ttfamily geometry\_option\_plasma} = 6: The plasma surface will be loaded from an ASCII file, specified by \parlink{shape\_filename\_plasma}. The first line of this file is ignored. The second line is an integer giving the number of Fourier modes
to read. The remaining lines contain $m$, $n$, $rmnc$, $zmns$, $rmns$, $zmnc$. \\

{\ttfamily geometry\_option\_plasma} = 7: The plasma surface and Bnorm information will be loaded from an ASCII file in FOCUS format, specified by \parlink{shape\_filename\_plasma}. For more information, please look here \url{https://princetonuniversity.github.io/FOCUS/rdsurf.pdf}.

}

\myhrule

\param{shape\_filename\_plasma}
{string}
{{\ttfamily ""}}
{Only when \parlink{geometry\_option\_plasma} is 6 or 7.}
{ASCII file from which to read in the plasma shape.}

\myhrule

\param{R0\_plasma}
{real}
{10.0}
{Only when \parlink{geometry\_option\_plasma} is 0 or 1.}
{Major radius of the plasma surface, when this surface is a plain circular torus.}

\myhrule

\param{a\_plasma}
{real}
{0.5}
{Only when \parlink{geometry\_option\_plasma} is 0 or 1.}
{Minor radius of the plasma surface, when this surface is a plain circular torus.}

\myhrule

\param{nfp\_imposed}
{integer}
{1}
{Only when \parlink{geometry\_option\_plasma} is 0 or 1.}
{When the plasma surface is a plain circular torus, only toroidal mode numbers that are a multiple of this parameter will be considered.
This parameter thus plays a role like \vmec's {\ttfamily nfp} (number of field periods),
and is used when {\ttfamily nfp} is not already loaded from a \vmec~file.}

\myhrule

\param{wout\_filename}
{string}
{{\ttfamily ""}}
{Only when \parlink{geometry\_option\_plasma} is 2, 3, or 4.}
{Name of the \vmec~{\ttfamily wout} output file which will be used for the plasma surface.
You can use either a \netCDF~or {\ttfamily ASCII} format file.}

\myhrule

\param{efit\_filename}
{string}
{{\ttfamily ""}}
{Only when \parlink{geometry\_option\_plasma} is 5.}
{Name of the {\ttfamily efit} output file which will be used for the plasma surface.}

\myhrule

\param{efit\_psiN}
{real}
{0.98}
{Only when \parlink{geometry\_option\_plasma} is 5.}
{Value of normalized poloidal flux at which to select a flux surface from the {\ttfamily efit} input file.
A value of 1 corresponds to the last closed flux surface, and 0 corresponds to the magnetic axis.}

\myhrule

\param{efit\_num\_modes}
{integer}
{10}
{Only when \parlink{geometry\_option\_plasma} is 5.}
{Controls the number of Fourier modes used to represent $R(\theta)$ and $Z(\theta)$ for the shape of
the plasma surface. Each of these functions will be expanded in a sum of functions $\sin(m\theta)$ and $\cos(m\theta)$,
where $m$ ranges from 0 to {\ttfamily efit\_num\_modes}$-1$.}

\myhrule

\section{Geometry parameters for the magnetization region}

For greater code consistency with standard REGCOIL, many variables describing the magnetization region include the word {\ttfamily coil}
even though they describe a region of magnetization rather than coils.

\myhrule

\param{geometry\_option\_coil}
{integer}
{0}
{Always}
{This option controls which type of geometry is used for the inner boundary surface of the magnetization region.\\

{\ttfamily geometry\_option\_coil} = 0: The coil surface will be a plain circular torus. The major radius will be the 
same as the plasma surface: either \parlink{R0\_plasma} if \parlink{geometry\_option\_plasma} is 0 or 1, or {\ttfamily Rmajor\_p} from the \vmec~{\ttfamily wout} file
if  \parlink{geometry\_option\_plasma} is 2.
     The minor radius will be \parlink{a\_coil}.\\

{\ttfamily geometry\_option\_coil} = 1: Identical to option 0, except the major radius of the magnetization surface will be set by \parlink{R0\_coil}.\\

{\ttfamily geometry\_option\_coil} = 2: The magnetization surface will computing by expanding the plasma surface uniformly by a distance \parlink{separation}. The expanded surface will be saved to a local file specified by \parlink{nescin\_filename}. \\

{\ttfamily geometry\_option\_coil} = 3: The magnetization surface will be the `coil' surface in the \nescoil~`nescin' input file specified by \parlink{nescin\_filename}. \\

{\ttfamily geometry\_option\_coil} = 4: Similar to option 2, except that the poloidal angle will be changed such that the arclength (with respect to $\theta)$ is independent of $\theta$ at each $\zeta$. The magnetization surface will computing by expanding the plasma surface uniformly by a distance \parlink{separation}. The expanded surface will be saved to a local file specified by \parlink{nescin\_filename}. \\
}

\myhrule

\param{R0\_coil}
{real}
{10.0}
{Only when \parlink{geometry\_option\_coil} is 1.}
{Major radius of the magnetization surface, when this surface is a plain circular torus.}

\myhrule

\param{a\_coil}
{real}
{1.0}
{Only when \parlink{geometry\_option\_coil} is 0 or 1.}
{Minor radius of the magnetization surface, when this surface is a plain circular torus.}


\myhrule

\param{separation}
{real}
{0.2}
{Only when \parlink{geometry\_option\_coil} is 2.}
{Amount by which the magnetization surface is offset from the plasma surface.}

\myhrule

\param{nescin\_filename}
{string}
{{\ttfamily "nescin.out"}}
{Only when \parlink{geometry\_option\_coil} is 2 (write to) or 3 (read from).}
{Name of a {\ttfamily nescin} file, of the sort used with the \nescoil~code.
If \parlink{geometry\_option\_coil}=3, the coil surface from
this file will be used as the magnetization surface for \regcoil. 
If \parlink{geometry\_option\_coil}=2, \regcoil~will save the uniform-offset surface it computes
into a file with this name.}

\myhrule

\param{mpol\_coil\_filter}
{integer}
{9999}
{Only when \parlink{geometry\_option\_coil} is 2, 3, or 4.}
{Terms in the Fourier series for $R(\theta,\zeta)$ and $Z(\theta,\zeta)$ describing the magnetization surface will be dropped if the poloidal mode number is larger than {\ttfamily mpol\_coil\_filter}.}

\myhrule

\param{ntor\_coil\_filter}
{integer}
{9999}
{Only when \parlink{geometry\_option\_coil} is 2, 3, or 4.}
{Terms in the Fourier series for $R(\theta,\zeta)$ and $Z(\theta,\zeta)$ describing the magnetization surface will be dropped if the toroidal mode number is larger than {\ttfamily ntor\_coil\_filter}. Specify 1, 2, 3, $\ldots$ rather than {\ttfamily nfp}, 2$\times${\ttfamily nfp}, 3$\times${\ttfamily nfp}, etc.}

\myhrule

\section{Parameters related to the thickness of the magnetization region}

\param{d\_option}
{string}
{{\ttfamily "uniform"}}
{Always}
{Determines how the thickness $d$ of the magnetization layer is computed.\\

{\ttfamily d\_option} = {\ttfamily uniform}: The magnetization layer will have a uniform thickness set by \parlink{d\_initial}. As a result, the 
magnetization will have nonuniform mangitude. This option is the fastest since only a single linear solve is required.\\

{\ttfamily d\_option} = {\ttfamily Picard}: Solve for the thickness $d$ such that the $s$-averaged $|M|$ is uniform and equal to \parlink{target\_mu0\_M}$/\mu_0$,
 using Picard iteration. 
This option is not recommended since convergence for this method is slower than with the {\ttfamily Anderson} option.\\

{\ttfamily d\_option} = {\ttfamily Anderson}: Solve for the thickness $d$ such that the $s$-averaged $|M|$ is uniform and equal to \parlink{target\_mu0\_M}$/\mu_0$, using Anderson-accelerated
fixed-point iteration \cite{Tonatiuh}.
}

\myhrule

\param{nd}
{integer}
{1}
{When \parlink{d\_option} is not "uniform".}
{Number of iterations of Picard or Anderson iteration that will be performed to determine the thickness $d$.}

\myhrule

\param{d\_initial}
{real}
{0.01}
{Always}
{When \parlink{d\_option}={\ttfamily "uniform"}, this parameter sets the uniform thickness of the magnetization region.
When \parlink{d\_option} is not {\ttfamily "uniform"}, this parameter sets the initial guess for the thickness of the magnetization region. Units = meters.}

\myhrule

\param{target\_mu0\_M}
{real}
{1.4}
{When \parlink{d\_option} is not ``uniform''.}
{Magnitude of the magnetization that will be used to determine the thickness of the magnetization layer. Units = Tesla.}

\myhrule

\param{sign\_normal}
{integer}
{1}
{Always}
{If equal to $+1$, the magnetization region is obtained from the magnetization boundary surface by extending in the direction 
$(\partial \vect{r}/\partial\zeta) \times (\partial \vect{r}/\partial\theta)$. If equal to $-1$, the magnetization region
is obtained by extending the boundary surface in the opposite direction.}

\myhrule

\param{Anderson\_depth}
{integer}
{2}
{When \parlink{d\_option}={\ttfamily "Anderson"}}
{Depth parameter for the Anderson iteration. See \cite{Tonatiuh}.}

\myhrule

\section{Parameters determining the normal magnetic field on the plasma surface}

There are three ways to specify $B_{normal}$: (1) by assuming the existence of TF coils that create a purely toroidal axisymmetric field
(set \parlink{include\_Bnormal\_from\_TF} = {\ttfamily .true.}), (2) by loading data from a bnorm file (set \parlink{load\_bnorm}={\ttfamily .true.}), or
(3) by loading data from a FOCUS file (set \parlink{geometry\_option\_plasma}=7).

\myhrule

\param{include\_Bnormal\_from\_TF}
{logical}
{{\ttfamily .false.}}
{Always}
{If true, the contribution to $B_{normal}$ from a perfectly axisymmetric field $\vect{B} = \vect{e}_{\zeta} \mu_0 G /(2\pi R)$
will be added to the total $B_{normal}$. The net poloidal current $G$ for this field is determined by \parlink{net\_poloidal\_current\_Amperes}
(if it is set to a nonzero value) or by the value corresponding to the vmec file \parlink{wout\_filename}.}

\myhrule

\param{net\_poloidal\_current\_Amperes}
{real}
{0.0}
{If \parlink{include\_Bnormal\_from\_TF}={\ttfamily .true.}}
{Net poloidal current $G$ linking the plasma poloidally. If this value is 0.0 and a VMEC wout file is used to determine the plasma surface shape,
then $G$ will be determined by the value corresponding to that VMEC configuration. If {\ttfamily net\_poloidal\_current\_Amperes} is nonzero,
this value will override the value from the VMEC configuration. This provides a method to scale the magnitude of $B$ for a VMEC configuration.}

\myhrule

\param{load\_bnorm}
{logical}
{\false}
{When \parlink{general\_option} is not 7.}
{Whether or not an output file from the \bnorm~code is to be loaded.
Set this option to \true~if there is significant current in the plasma,
meaning the coils will need to cancel the associated magnetic field component normal
to the target plasma surface.
}

\myhrule

\param{bnorm\_filename}
{string}
{{\ttfamily ""}}
{When \parlink{geometry\_option\_plasma} is not 7 and \parlink{load\_bnorm}={\ttfamily .true.}.}
{Output file from the \bnorm~code which contains the magnetic field normal to the target
plasma surface associated with current inside the plasma and with coils such as planar TF coils.}

\myhrule

\section{Parameters related to the regularization weight}

\param{lambda\_option}
{string}
{"scan"}
{Always}
{Determines how values of the regularization parameter $\lambda$ are chosen.\\

{\ttfamily lambda\_option = "single"}: Use only a single value, \parlink{lambda\_single}.\\

{\ttfamily lambda\_option = "scan"}: Use a range of values of $\lambda$, determined by \parlink{lambda\_min}, \parlink{lambda\_max}, and \parlink{nlambda}.\\

{\ttfamily lambda\_option = "search"}: Search for a value of the regularization weight such that
a certain target is met. The target is chosen using \parlink{target\_option}.
}

\myhrule

\param{Nlambda}
{integer}
{4}
{Only when \parlink{lambda\_option} = {\ttfamily "scan"} or {\ttfamily "search"}.}
{When \parlink{lambda\_option}={\ttfamily "scan"}, {\ttfamily Nlambda} is the number of values of $\lambda$ for which the problem is solved.
When \parlink{lambda\_option}={\ttfamily "search"}, {\ttfamily Nlambda} is the upper limit on the number of values of $\lambda$ for which the problem is solved.}

\myhrule

\param{lambda\_max}
{real}
{1.0e-13}
{Only when \parlink{lambda\_option} = {\ttfamily "scan"}.}
{Maximum value of $\lambda$ for which the problem is solved.}

\myhrule

\param{lambda\_min}
{real}
{1.0e-19}
{Only when \parlink{lambda\_option} = {\ttfamily "scan"}.}
{Minimum nonzero value of $\lambda$ for which the problem is solved.
Note that the problem is always solved for $\lambda=0$ in addition to
the nonzero values.}

\myhrule

\param{target\_option}
{string}
{{\ttfamily "max\_M"}}
{Only when \parlink{lambda\_option} = {\ttfamily "search"}.}
{Controls which quantity is targeted to determine $\lambda$:\\

{\ttfamily target\_option = "max\_M"}: Search for the $\lambda$ value such that the maximum
magnetization in the magnetization volume equals \parlink{target\_value}.\\

{\ttfamily target\_option = "chi2\_M"}: Search for the $\lambda$ value such that $\chi^2_M$ equals \parlink{target\_value}.\\

{\ttfamily target\_option = "max\_Bnormal"}: Search for the $\lambda$ value such that the maximum
$\vect{B}\cdot\vect{n}$ over the plasma surface equals \parlink{target\_value}.\\

{\ttfamily target\_option = "chi2\_B"}: Search for the $\lambda$ value such that $\chi^2_B$ equals \parlink{target\_value}.\\

{\ttfamily target\_option = "rms\_Bnormal"}: Search for the $\lambda$ value such that the root-mean-square value of $\vect{B}\cdot\vect{n}$, i.e.
$\left( \int d^2a\; B_n^2 \right)^{1/2}$ (where the integral is over the plasma surface) equals \parlink{target\_value}.

}

\myhrule

\param{target\_value}
{real}
{8.0e6}
{Only when \parlink{lambda\_option} = {\ttfamily "search"}.}
{The value of the quantity specified by \parlink{target\_option} that the code will attempt to match
by varying $\lambda$.
}

\myhrule

\param{lambda\_search\_tolerance}
{real}
{1.0e-5}
{Only when \parlink{lambda\_option} = {\ttfamily "search"}.}
{Relative tolerance for the lambda root-finding.}

\myhrule

\section{Parameters related to saving mgrid files}

\param{write\_mgrid}
{logical}
{{\ttfamily .false.}}
{Always}
{If true, an mgrid file (for use with free-boundary VMEC) will be saved using the last configuration computed, i.e. the last values of $\lambda$ and $d$.
Typically you would set this parameter to true only when setting \parlink{lambda\_option}={\ttfamily "single"}, so you know which value of lambda will be used.}

\myhrule

\param{mgrid\_rmin}
{real}
{0.9}
{Only when \parlink{write\_mgrid} = {\ttfamily .true.}.}
{Minimum major radius at which $\vect{B}$ will be saved in the mgrid file, corresponding to the usual {\ttfamily rmin} parameter of {\ttfamily makegrid}.}

\myhrule

\param{mgrid\_rmax}
{real}
{2.0}
{Only when \parlink{write\_mgrid} = {\ttfamily .true.}.}
{Maximum major radius at which $\vect{B}$ will be saved in the mgrid file, corresponding to the usual {\ttfamily rmax} parameter of {\ttfamily makegrid}.}

\myhrule

\param{mgrid\_zmin}
{real}
{-0.75}
{Only when \parlink{write\_mgrid} = {\ttfamily .true.}.}
{Minimum cylindrical coordinate $z$ at which $\vect{B}$ will be saved in the mgrid file, corresponding to the usual {\ttfamily zmin} parameter of {\ttfamily makegrid}.}

\myhrule

\param{mgrid\_zmax}
{real}
{0.75}
{Only when \parlink{write\_mgrid} = {\ttfamily .true.}.}
{Maximum cylindrical coordinate $z$ at which $\vect{B}$ will be saved in the mgrid file, corresponding to the usual {\ttfamily zmax} parameter of {\ttfamily makegrid}.}

\myhrule

\param{mgrid\_ir}
{integer}
{99}
{Only when \parlink{write\_mgrid} = {\ttfamily .true.}.}
{Number of grid points in major radius at which $\vect{B}$ will be saved in the mgrid file, corresponding to the usual {\ttfamily ir} parameter of {\ttfamily makegrid}.}

\myhrule

\param{mgrid\_jz}
{integer}
{101}
{Only when \parlink{write\_mgrid} = {\ttfamily .true.}.}
{Number of grid points in the cylindrical coordinate $z$ at which $\vect{B}$ will be saved in the mgrid file, corresponding to the usual {\ttfamily jz} parameter of {\ttfamily makegrid}.}

\myhrule

\param{mgrid\_kp}
{integer}
{20}
{Only when \parlink{write\_mgrid} = {\ttfamily .true.}.}
{Number of grid points (per field period) in toroidal angle at which $\vect{B}$ will be saved in the mgrid file, corresponding to the usual {\ttfamily kp} parameter of {\ttfamily makegrid}.}



